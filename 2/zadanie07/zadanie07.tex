\documentclass[a4paper,11pt]{article}
\usepackage[polish]{babel}
\usepackage[utf8]{inputenc}
\usepackage[T1]{fontenc}
\usepackage{anysize}
\usepackage{times}
\usepackage{multirow}
\usepackage{amsmath}

%\marginsize{left}{right}{top}{bottom}
\marginsize{3cm}{3cm}{3cm}{3cm}
\sloppy

\begin{document}

    Istnieje ścisły związek między rozkładem macierzy \(A\) na macierze \(L\) i \(U\) a metodą eliminacji Gaussa.
    Mozna wykazać, że elementy kolejnych kolumn macierzy \(L\) są równe współczynnikom przez które mnożone
    są w kolejnych krokach wiersze układu równań celem dokonania eliminacji niewiadomych w odpowiednich
    kolumnach. Natomiast macierz \(U\) jest równa macierzy trójkątnej uzyskanej w eliminacji Gaussa.

    \[
        [A|b] =
        \begin{bmatrix}
            2 & 2 & 4 & 4 \\
            1 & 2 & 2 & 4 \\
            1 & 4 & 1 & 1 \\
        \end{bmatrix} =
        \begin{bmatrix}
            2 & 2 & 4 & 4 \\
            0 & 1 & 0 & 2 \\
            0 & 3 & -1 & -1 \\
        \end{bmatrix} =
        \begin{bmatrix}
            2 & 2 & 4 & 4 \\
            0 & 1 & 0 & 2 \\
            0 & 0 & -1 & -7 \\
        \end{bmatrix}
    \]
    \begin{sloppypar}
    \centerline{
        ${
            L = \begin{bmatrix}
                1 & 0 & 0 \\
                \frac{1}{2} & 1 & 0 \\
                \frac{1}{2} & 3 & 1 \\
            \end{bmatrix}
            U = \begin{bmatrix}
                2 & 2 & 4 & 4 \\
                0 & 1 & 0 & 2 \\
                0 & 0 & -1 & -7 \\
            \end{bmatrix}
        }$
    }
    \end{sloppypar}

\end{document}
