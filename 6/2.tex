\documentclass[12pt,leqno]{article}
\usepackage[a4paper,top=3cm,bottom=3cm,left=3cm,right=3cm]{geometry}
\usepackage{amssymb}
\usepackage{amsmath}
\usepackage[T1]{fontenc}
\usepackage[utf8]{inputenc}
\usepackage[english]{babel}
\usepackage{fancyhdr}
\usepackage[center]{titlesec}
\usepackage{indentfirst}

\setcounter{page}{110}
\setcounter{section}{4}
\setcounter{subsection}{10}

\numberwithin{equation}{section}

\pagestyle{fancy}
\fancyhf{}
\renewcommand{\headrulewidth}{0pt}
\chead{\tiny{4. ELLIPTIC PDES}}
\lhead{\tiny\thepage}

\makeatletter
\renewcommand*{\@cite@ofmt}{\bfseries\hbox}
\makeatother

\begin{document}

\hfill\begin{minipage}{\dimexpr\textwidth-2cm}
In that case, the solutions are
$$u = \sum_{\lambda_n \neq \lambda} \frac{\left(f, \phi_n\right)}{\lambda_n - \lambda} \phi_n + \sum_{n=M}^{N} c_n \phi_n$$

where $\left\lbrace c_M,\dots,c_N\right\rbrace$ are arbitrary real constants.
\end{minipage}

\subsection{Interior regularity}

\setcounter{equation}{33}

Roughly speaking, solutions of elliptic PDFe are as smooth as the data allows.
For boundary value problems, it is convenient to consider the regularity of the
solution in the interior of the domain and near the boundary speparately. We begin
by studying the interior regularity of solutions. We follow closely the presentation
in \cite{SomePresentation}.

To motivate the regularity theory, consider the following simple \emph{a priori} esti\-1mate
for the Laplacian. Suppose that $u \in C_c^{\infty}(\mathbb{R}^n)$. Then, integrating by parts
twice, we get
\begin{equation*}
\begin{split}
\int (\Delta u)^2 dx & = \sum_{i,j=1}^{n} \int (\partial_{ii}^2 u) (\partial_{jj}^2 u) \ dx \\
& = - \sum_{i,j=1}^{n} \int (\partial_{iij}^3 u) (\partial_{j}^2 u) \ dx \\
& = \sum_{i,j=1}^{n} \int (\partial_{ij}^2 u) (\partial_{ij}^2 u) \ dx \\
& = \int \left|D^2 u\right|^2 \ dx.
\end{split}
\end{equation*}
Hence, if $- \Delta u = f$, then
$$ \lVert D^2 u \rVert_{L^2} = \lVert f \rVert_{L^2}^2.$$
Thus, w can control the $L^2$-norm of all second derivatives of $u$ by the $L^2$-norm
of the Laplacian of $u$. This estimate suggest that we should have $u \in H_{loc}^2$ if
$f, u \in L^2$, as is in fact true. The above computation is, however, not justified for
weak solutions that belong to $H^1$; as far as we know from previous existence
theory, such solutions may not event posses second-order weak derivatives.

We will consider a PDE
\begin{align}
\label{eq:firstEq}
Lu = f && \text{in } \Omega
\end{align}
where $\Omega$ is an open set in $\mathbb{R}^n, f \in L^2(\Omega)$, and $l$ is a uniformly elliptic of the form
\begin{equation}
\label{eq:secondEq}
Lu = - \sum_{i,j=1}^{n} \partial_i (a_{ij} \partial_j u).
\end{equation}
It is straightforward to extend the proof of the regularity theorem to uniformmly
elliptic operations thath contain lower-order terms \cite{SomePresentation}.

A function $u \in H^i(\Omega)$ is a weak solution of (\ref{eq:firstEq})--(\ref{eq:secondEq}) if
\begin{align}
\label{eq:thirdEq}
a(u,v) = (f,v) && \text{for all } v \in H_0^1(\Omega),
\end{align}

\begin{thebibliography}{x}
\makeatletter
\addtocounter{\@listctr}{8}
\makeatother

\bibitem{SomePresentation}
Some Presentation
\end{thebibliography}

\end{document}
