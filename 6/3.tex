\documentclass[a4paper,11pt,twoside]{article}
\usepackage{amssymb}
\usepackage{amsthm}
\usepackage[polish]{babel}
\usepackage[utf8]{inputenc}
\usepackage[T1]{fontenc}
\usepackage{geometry}
\usepackage{xcolor}
\usepackage{times}
\usepackage{graphicx}
\usepackage{anysize}
\usepackage{enumerate}
\usepackage{listings}
\usepackage{algorithm}
\usepackage{algpseudocode}
\usepackage{amsfonts}
\usepackage{anysize}
\usepackage{amsmath}
\usepackage{longtable}
\usepackage{fancyhdr}
\usepackage{makeidx}
\usepackage{array}
\usepackage{tikz}
\usetikzlibrary{calc,through,backgrounds,positioning,fit}
\usetikzlibrary{shapes,arrows,shadows,calendar}
\usepgflibrary{arrows}
\usepackage{newunicodechar}
\usepackage[export]{adjustbox}

%\newunicodechar{fi}{fi}


\marginsize{1.5cm}{1.5cm}{1.5cm}{1.5cm}
\sloppy
\renewcommand{\headrulewidth}{0pt}

\pagestyle{fancy}
\rhead{CHAPTER 14. COMPTON SCATTERING}
\lhead{172}
\cfoot{}
\setcounter{section}{13}



\begin{document}
    %Marcin Flis 17:30 27.05
    \section{a}

    \begin{equation}
    \label{14.21}
    P_{-} = \langle h\nu \rangle \frac{d N}{dt} = c\sigma_{T} u
    \tag{14.21}
    \end{equation}

    \begin{equation}
    \label{14.7}
    P_{-} = \langle h\nu \rangle \frac{d N}{dt} = c\sigma_{T} u
    \tag{14.7}
    \end{equation}

    \begin{equation}
    \label{14.19}
    P_{-} = \langle h\nu \rangle \frac{d N}{dt} = c\sigma_{T} u
    \tag{14.19}
    \end{equation}

    \newpage

    Now Thomson scattering is independent of frequency of photon, so it follows that the rate at which
    energy is scattered out of radiation field is given by multiplying $dN/dt$ by the mean energy per
    photon, to give
    \begin{equation}
    \label{14.24}
    P_{-} = \langle h\nu \rangle \frac{d N}{dt} = c\sigma_{T} u
    \tag{14.24}
    \end{equation}

    Note that this is precisely the rate at which energy is scattered for a stationary electron. This is
    something of a coincidence since we can see from (\ref{14.21}) that the radiation field by moving
    electron do \textit{not} have an isotrophic distribution in the lab-frame, rather they have a $( 1-
    \beta \cos \theta)$ distribution, so the photons propagating in the direction opposite to the electron
    are more likely to be scattered.

    Combining $P_{+} $ from (\ref{14.19}) and $ P_{-}$ form (\ref{14.24}) gives the net
    \textit{inverse Compton power} for 1 electron of $ P = P_+ - P_- $ or

    \begin{equation}
    \label{14.25}
    P = \frac{4}{3} \beta^2 \gamma^2 c \sigma_{T} u
    \tag{14.25}
    \end{equation}

    and the total energy transfer rate per unit volume is given by mulitiplying this by the elctron density,
    or more generally by the distrobution function $ n(\textbf{r},E)$ and intergrating over energy.

    Equation (\ref{14.25}) is remarkably simple, and also ramerkably similar to the synchrotron power
    and bremsstrahlung power, for reasons already disscussed.

    Interstingly, for low velocities, the Compton power is quadratic in the velocity. There is no first
    order-effect, since a scatterings may increase or decrease the photon energy.

    \setcounter{subsection}{3}
    \subsection{Compton vs Inverse Compton Scattering}
    Equation (\ref{14.25}) is supposedly valid for all electron eneregies, and is clealry always positive.
    However, this does not make sense. For cold electrons, Compton scattering result in a loss of energy
    for the elsctrons vis the recoil, which was ignored in deriving (\ref{14.25}).

    For the low eneregy electrons (with $ v/c = \beta \ll 1$), the radiation in the electron frame is very
    nearly isotropic ($\delta \nu / \nu \sim \beta \ll 1$, so consequently the variation of intensity $\delta
    I/I \sim \beta \ll 1 $ and is also small, so we can incorpotate the effect of recoil by simply
    subtracting the meran photon energy loss givern by (\ref{14.7}).

    For $\upsilon \ll c $ mean rate of energy transfer is given by $ dE/ dt \backsimeq (4/3)c \sigma_{T}
    u \upsilon^2 / c^2 $ while the rate of scatterings is $dN/dt = c \sigma_{T} n = c \sigma_{T} u /
    \langle \epsilon \rangle $ so the mean photon eneregy gain per collision (neglecting recoil) is $
    \langle \Delta \epsilon \rangle / \langle \epsilon \rangle = (4/3)(v/c)^2 $, and if $v \ll c $ that is
    approximately equal to the mean \textit{frractional} energy gain $ \langle \Delta \epsilon /
    \epsilon \rangle = (4/3)(v/c)^2 $ For a thermal distribution of electrons this becomes $\langle \Delta
    \epsilon / \epsilon \rangle = 4kT/ mc^2 $. The mean fractional eneregy loss due to recoil if from
    (\ref{14.7}) $ \Delta \epsilon / \epsilon = \epsilon/ Mc^2 $ so combining these gives

    \begin{equation}
    \label{14.26}
    \left\langle \frac{\Delta \epsilon}{\epsilon} \right\rangle = \frac{4kT - h\omega}{m c^2}
    \tag{14.26}
    \end{equation}

    It $\epsilon > 4kT $ then there is net transfer of energy to the electrons and \textit{vice versa}.


    \subsection{ The Compton $y$- Parameter}

    The \textit{ Compton y-parameter } is defined as

    \begin{equation*}
    \label{14}
    y \equiv \left\langle \frac{\Delta \epsilon}{\epsilon} \right\rangle \times \langle \text{number of
        scatterings} \rangle
    \end{equation*}

    \begin{itemize}
        \item In a system with $y$ much less (greater) then unity spactrum will be little (strongly)
        affected by scattering(s).
        \item In computing $y$ it is usual to either use the non relativistic expression (\ref{14.26})
        or the higlhy relativistic limit $ \left \langle \Delta \epsilon / \epsilon \right\rangle \simeq 4
        \gamma^2 /3 $.
        \item The mean number of scatterings is given by $ \text{max}(\tau,\tau^2)$.
    \end{itemize}

\end{document}
