\documentclass[a4paper,11pt]{article}
\usepackage[polish]{babel}
\usepackage[utf8]{inputenc}
\usepackage[T1]{fontenc}
\usepackage{graphicx}
\usepackage{anysize}
\usepackage{enumerate}

%\marginsize{left}{right}{top}{bottom}
\marginsize{2.5cm}{2.5cm}{2.5cm}{2.5cm}
\sloppy



\begin{document}

Niezależnie od systematyki przyjęto tutaj podział wszystkich ptaków na trzy grupy według ich użyteczności gospodarczej: ptaki łowne, które w~okresie lęgów znajdują się pod tzw. ochroną okresową, ptaki objęte całoroczną ochroną, zwaną gatunkową, ptaki, które w~ciągu całego roku w~ogóle nie podlegają ochronie.

\begin{itemize}
\item Niezależnie od systematyki przyjęto tutaj podział wszystkich ptaków na trzy grupy według ich użyteczności gospodarczej: ptaki łowne, które w~okresie lęgów znajdują się pod tzw. ochroną okresową, ptaki objęte całoroczną ochroną, zwaną gatunkową, ptaki, które w~ciągu całego roku w~ogóle nie podlegają ochronie.

Niezależnie od systematyki przyjęto tutaj podział wszystkich ptaków na trzy grupy według ich użyteczności gospodarczej: ptaki łowne, które w~okresie lęgów znajdują się pod tzw. ochroną okresową, ptaki objęte całoroczną ochroną, zwaną gatunkową, ptaki, które w~ciągu całego roku w~ogóle nie podlegają ochronie.

\begin{enumerate}[{[}i.{]}]
	\item Ala
	\item Ela
	\item Ola
	\item Ala
	\item Ela
	\item Ola
\end{enumerate}

\item Niezależnie od systematyki przyjęto tutaj podział wszystkich ptaków na trzy grupy według ich użyteczności gospodarczej: ptaki łowne, które w~okresie lęgów znajdują się pod tzw. ochroną okresową, ptaki objęte całoroczną ochroną, zwaną gatunkową, ptaki, które w~ciągu całego roku w~ogóle nie podlegają ochronie.

\item Niezależnie od systematyki przyjęto tutaj podział wszystkich ptaków na trzy grupy według ich użyteczności gospodarczej: ptaki łowne, które w~okresie lęgów znajdują się pod tzw. ochroną okresową, ptaki objęte całoroczną ochroną, zwaną gatunkową, ptaki, które w~ciągu całego roku w~ogóle nie podlegają ochronie.
\end{itemize}



\begin{itemize}
	\item Ala
	\item Ela
	\item Ola
\end{itemize}

\begin{enumerate}
\item Ala

\item Ula

\begin{itemize}
\item kot
\item pies
\end{itemize}

\item Ela

\item Ola
\end{enumerate}


\begin{description}
\item[ptaki łowne] -- ptaki, które w okresie lęgów znajdują się pod tzw. ochroną okresową

\item[aaaa] -- ptaki \textit{objęte całoroczną ochroną}, zwaną gatunkową,

\textit{\textbf{ala ma kota}}


\end{description}

Niezależnie od systematyki \%przyjęto tutaj podział wszystkich ptaków na trzy grupy według ich użyteczności gospodarczej:

\begin{enumerate}[A.]
\item ptaki łowne, które w okresie lęgów znajdują się pod tzw. ochroną okresową,
\item ptaki objęte całoroczną ochroną, zwaną gatunkową,
\item ptaki, które w ciągu całego roku w ogóle nie podlegają ochronie.

\begin{enumerate}[{\theenumi}.a)]
\item Ela

\item Ola
\end{enumerate}


\item ptaki łowne, {\LARGE które w okresie} lęgów znajdują się pod tzw. ochroną okresową,
\item ptaki objęte całoroczną ochroną, zwaną gatunkową,
\item ptaki, które w ciągu całego roku w ogóle nie podlegają ochronie.
\end{enumerate}
   



\end{document}
